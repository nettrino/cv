%-------------------------------------------------------------------------------
%	SECTION TITLE
%-------------------------------------------------------------------------------

%-------------------------------------------------------------------------------
%	CONTENT
%-------------------------------------------------------------------------------
\cvsection{Research Experience}

\cvsubsection{Software Testing}

\begin{cvparagraph}

Automatically discovering bugs in security-sensitive code is hard.  In my
recent work, I designed and developed \nezha~[\ref{itm:nezha}],
an evolutionary-based,
input-format-agnostic fuzzer which focuses on semantic bugs. This fuzzer
utilizes novel black-box and gray-box input generation mechanisms and has found
multiple bugs in security critical applications like SSL libraries (\eg
LibreSSL, GnuTLS, WolfSSL) and anti-virus software, reporting semantic bugs
missed by state-of-the-art fuzzers like
\href{http://lcamtuf.coredump.cx/afl/}{AFL} and
\href{http://llvm.org/docs/LibFuzzer.html}{libFuzzer}.

When auditing an application for bugs and vulnerabilities, it is vital to
determine what constitutes a true bug, rather than a developer-intended
violation of a specification. In previous work, I co-developed
\intflow [\ref{itm:intflow}], a compiler extension which uses taint
analysis to report integer errors, taking into account common developer practices
to reduce false positives. \intflow achieves 89\% reduction in
false positives compared to the Clang-based Integer Overflow Checker (IOC).

My latest research in this field attempts to enhance static analysis tools to
further improve fuzzing, as well as to detect new types of bugs such as
resource exhaustion bugs~[\ref{itm:slowfuzz}].
\end{cvparagraph}

\cvsubsection{Software Hardening}

\begin{cvparagraph}

Applications and Operating Systems can be separated into discreet parts, each
facing different types of threats. In the past, I have worked on software
hardening defenses, aiming at efficiently addressing such security issues. One
such example of my work is \dynaguard [\ref{itm:dynaguard}], which secures
canary-based protections against brute-force attacks. \dynaguard comes in two
flavors, a compiler-based, inducing a 1.2\% overhead, and a version
implemented using Dynamic Binary Instrumentation (DBI), which can applied
as-is to black-box binaries.

Another defense in this space which I have co-developed is, \krx a practical
and comprehensive kernel hardening scheme that protects the Linux kernel
from code-reuse attacks [\ref{itm:krx}]. The above scheme combines an
execute-only memory principle with code diversification defenses, and can
benefit from hardware support (\eg MPX on modern Intel CPUs) to optimize
performance.

\end{cvparagraph}

\cvsubsection{Privacy}

\begin{cvparagraph}

Modern users face a multitude of privacy threats. In my research, I seek to
evaluate the privacy guarantees that modern services provide, as well as to
develop tools that enable users to use these services in a transparent,
privacy-preserving manner. In
past work, I co-performed a formal analysis of the defenses that major
Location Based Services (LBSes)[\ref{itm:istalker},\ref{itm:istalker_tops}]
deploy against location disclosure attacks. I also co-developed novel attacks
to bypass these defenses, across all major services, like Facebook, Foursquare
and others, and was the primary author of an open-source Python
framework for auditing Location Based Services with respect to their
location privacy guarantees.

With respect to Web transparency, in previous work I co-developed XRay
[\ref{itm:xray}], a framework that utilizes differential correlation to
determine which particular actions of users (emails, YouTube views and
clicks \etc) are linked with their targeting by online services.

My latest work in this field involves developing new methods to fingerprint
users, using behavioral patterns derived form their online presence, and
implementing viable, real-world defenses for the users' protection against
such attacks.

\end{cvparagraph}

\cvsubsection{Web Security}
\begin{cvparagraph}

In 2014 I developed a compiler extension offering SQL injection protection to
C/C++ applications. Currently, I am co-developing a symbolic execution
engine for PHP, which operates at the PHP interpreter level and fully supports
database operations and loose types. Finally, some of my
current research in the field involves examining the discrepancies in the
way the various entities of the Web ecosystem handle content, developing novel
attacks due to content handling confusion and suggesting real-world defenses
for the presented attacks.
\end{cvparagraph}

